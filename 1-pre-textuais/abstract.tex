Context: the demand for fast and efficient software development processes is evident in the software market. A key factor in improving this aspect is the effectiveness in communicating requirements. There are factors that positively or negatively influence this communication. Objective: this work aims to perform a study to identify and classify  factors that influence the communication in the requirements engineering process. Method: in order to reach this goal, a systematic review of the literature was carried out, It returned 860 studies from four databases and 94 were accepted for data extraction. Results: from the analysis of the returned studies, we verified the existence of 38 factors, which can be positive or negative, that act in the process of requirements communication in different aspects. Conclusions: the understanding and analysis of these factors that affect the requirements communication is decisive in order to obtain an effective development process. Thus, knowing these factors is relevant to industry because it allows project managers and analysts to develop strategies to mitigate negative factors and maximize the effects of positive factors. In addition, it assists in the organizational context of mature companies that are able to serve multiple clients, but fail to meet exceptional satisfaction of small organizations. They can also be benefited by optimizing the development process to a comprehensive scope or a greater customer demand.
% Separe as Keywords por ponto

\keywords{Requirements engineering. Requirements communication. Communication factors. Software Development.}