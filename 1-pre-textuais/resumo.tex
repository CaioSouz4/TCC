Contexto: a demanda por processos de desenvolvimento de softwares cada vez mais rápidos e eficientes é evidente no mercado de software. Um fator preponderante na melhoria desse aspecto é a eficácia na comunicação de requisitos. Existem fatores que influenciam, positivamente ou negativamente, essa comunicação de requisitos. Objetivo: este trabalho visa realizar um estudo para identificar e classificar fatores influentes na comunicação no processo de engenharia de requisitos. Método: Para alcançar esse objetivo, foi realizada uma revisão sistemática da literatura que retornou 860 trabalhos de quatro bases de dados sendo 94 aceitos para extração de dados.
Resultados: a partir da análise dos estudos retornados, verificou-se a existência de 38 fatores, que podem ser positivos ou negativos, que atuam no processo de comunicação de requisitos em diferentes aspectos. Conclusões: a compreensão e análise desses fatores que afetam a comunicação de requisitos é determinante para obter um processo de desenvolvimento eficaz. Sendo assim, o conhecimento dos fatores é relevante para a indústria, pois permite que os gerentes de projetos e analistas elaborem estratégias para mitigar os fatores negativos e maximizem os efeitos dos fatores positivos. Além disso, auxilia no contexto organizacional de empresas maduras que possuem melhor capacidade em atender vários clientes, mas falham em atender com excepcional satisfação como as organizações pequenas. Estas também podem se beneficiar ao otimizar o processo de desenvolvimento em um escopo abrangente ou uma demanda maior de clientes.

 


 

% Separe as palavras-chave por ponto
\palavraschave{Engenharia de requisitos. Comunicação de requisitos. Fatores de comunicação. Desenvolvimento de software.}