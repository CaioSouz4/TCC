
\chapter{PROTOCOLO DA REVISÃO SISTEMÁTICA}
\label{sec:protocolo}       

O protocolo de uma revisão sistemática da literatura especifica os métodos que serão empregados para realizar a revisão. De acordo com \cite{kitchenham}, um protocolo bem definido é necessário para reduzir a possibilidade de viés do pesquisador. Por exemplo, sem um protocolo, é possível que a seleção de
estudos ou a análise seja conduzida segundo as expectativas do pesquisador. Nas próximas seções, são descritos os campos desse protocolo.

A definição do protocolo utilizado nessa pesquisa seguiu as \textit{guidelines} de \cite{kitchenham}. O protocolo, que foi definido utilizando a ferramenta \textit{Parsifal}, é detalhado nas próximas seções.

\section{Definição do objetivo} 
    
O objetivo desta revisão é identificar estudos que descrevam fatores que influenciam a comunicação de requisitos no processo de desenvolvimento de software.
%diretamente ou indiretamente na comunicação de requisitos.


\section{Perguntas de Pesquisa} \label{sec:RQs}


O critério PICOC (População, Intervenção, Comparação, Resultado e Contexto), do inglês \emph{Population, Intervention, Comparison, Outcome, Context}, foi utilizado para direcionar a definição das perguntas de pesquisa. 

\begin{itemize}
\item\textbf{População:} Publicações revisadas aos pares que descrevem fatores que influenciam a comunicação de requisitos no processo de desenvolvimento de software.
\item\textbf {Intervenção:} Coletar evidências empíricas em relação aos fatores que influenciam a comunicação de requisitos.
\item\textbf {Comparação:} Não se aplica, pois os estudos primários não serão comparados. 
\item\textbf {Resultados:} Respostas para as perguntas de pesquisa.
\item\textbf {Contexto:} Engenharia de Software desde que \emph{stakeholders} comuniquem requisitos.
\end{itemize}

Considerando o objetivo dessa revisão e o critério PICOC, pretende-se responder as perguntas de pesquisa apresentadas na Tabela \ref{tab:rqs}.

        \begin{table}[h!]
        \centering
        \caption{Perguntas de pesquisa.}
        \begin{tabular}{p{12cm}}
        
        \hline
        
P1. Quais são os stakeholders envolvidos? \\\hline
P2. Quais são os problemas de comunicação reportados no estudo?\\\hline
P3. Quais são os fatores de comunicação apontados no estudo?\\\hline
P4. Qual o impacto do fator de comunicação?\\\hline
P5. O fator interfere em qual aspecto do processo de comunicação?\\\hline


        \label{tab:rqs}
        \end{tabular}
         \legend{Fonte: Elaborado pelo autor}
        \end{table}        

\section{String de Busca}

A \emph{string} de busca foi definida considerando os principais termos dos conceitos sob investigação. As palavras-chave são listadas na Tabela \ref{tab:keywords}.

        \begin{table}[h!]
        \centering
        \caption{Palavras chave.}
        \label{tab:keywords}
        \begin{tabular}{p{5cm}p{9cm}}
        \hline
        \textbf{Conceito}  & \textbf{Palavras-chave}   \\ \hline
        Problemas de comunicação                               & Communication Challenges, Communication Risk, Requirements Issues, Requirements communication       \\ \hline
        Fatores & Factor, Action, Characteristic, Feature, Practice \\ \hline
        Requirement &Requirement, Functional Requirement, Functionality \\\hline
        
        \end{tabular}
         \legend{Fonte: Elaborado pelo autor}
        \end{table}
  
        Pesquisas piloto foram realizadas de maneira iterativa para refinar a \emph{string} de busca. Foram excluídas palavras-chave cuja inclusão não retornou documentos adicionais nas pesquisas automáticas. Após várias iterações, a seguinte \emph{string} foi usada para pesquisar nas palavras-chave, título, resumo e texto completo das publicações:

\emph{        
(“communication challenge” OR “communication problem” OR “communication
risk” OR “requirements issue” OR “requirements communication”) AND (“factor” OR “action”
OR “characteristic” OR “feature” OR "practice”) AND (“software engineering”) AND (“software
development”) AND (“requirement” OR “functional requirement” OR “functionality”)
}
\newpage

\section{Bases de dados}\label{sec:bases}

As bases de dados que foram utilizadas para a seleção dos artigos são descritas na Tabela \ref{tab:bases}.


\begin{table}[h]
\centering
\caption{Bases utilizadas na pesquisa.}
\begin{center}
\scriptsize
\label{base_de_pesquisa}
\begin{tabular}{p{5.0cm}p{5.0cm}}
\hline

\textbf{Nome}
&\textbf{URL}\\

\hline
ACM Digital Library &	http://portal.acm.org\\ 
IEEE Digital Library &	http://ieeexplore.ieee.org\\ 
Science@Direct &	http://www.sciencedirect.com \\	 
Scopus &	http://www.scopus.com \\
\hline
 
\end{tabular} 

\legend {\fontsize{10}{12}\selectfont {Fonte: Autor.}}
\end{center}
\label{tab:bases}
\end{table}


       \section{Critérios de seleção}
        \label{sec:criterios}
        
            Os critérios de seleção foram utilizados para determinar se um trabalho deve ou não ser incluído na revisão. Esses critérios estão descritos na Tabela \ref{tab:criterios}.
        
\begin{comment}

\begin{itemize}
            \item Estudos primários.
               \item Estudos publicado em qualquer ano até julho de 2018.
               \item Estudos que abordam nos objetivos comunicação de requisitos. 
               \item Estudos que relacionam requisitos e comunicação.
              
           \end{itemize}
    
            Da mesma forma, será adotado os seguintes critérios para exclusão:
      
            \begin{itemize}
                \item Artigos curtos (\emph{short papers}) com menos de quatro páginas.
                \item Estudos duplicados.
                \item Estudos incompletos.
                \item Estudos secundários.
                \item Estudos redundantes da mesma autoria.
                \item Estudos claramente irrelevantes para a pesquisa, levando em consideração as questões de pesquisa.
                \item Estudos cujo foco não se relaciona a comunicação de requisitos.
                \item Estudos cujo texto não esteja disponível.
                \item Literatura cinza (teses, dissertações, monografias, etc).
                \item Estudos não escritos em inglês.
            \end{itemize}
            
\end{comment}            
        

        \begin{table}[h!]
        \centering
        \caption{Critérios de inclusão e exclusão.}
        \label{tab:criterios}
        \footnotesize
        \begin{tabular}{p{6cm}p{6cm}}
        \hline
        \textbf{Critérios de inclusão}  & \textbf{Critérios de exclusão}   \\ \hline
        Estudos primários                               & Estudos curtos (\emph{short papers}) com menos de quatro páginas.       \\ \hline
        Estudos publicados em qualquer ano até julho de 2018. & Estudos duplicados. \\ \hline
        Estudos que abordam nos objetivos comunicação de requisitos.  &Estudos incompletos. \\\hline
        Estudos que relacionam requisitos e comunicação.  & Estudos secundários. \\\hline
        
        Estudos que respondam alguma pergunta de pesquisa.   & Documento redundante da mesma autoria.\\\hline
                                                         & Estudos claramente irrelevantes para a pesquisa, levando em consideração as questões de pesquisa.\\\hline
                                                         & Estudos cujo foco não relaciona-se a comunicação de requisitos.\\\hline
                                                         & Estudos cujo texto completo não esteja disponível.\\\hline
                                                         & Literatura cinza (teses, dissertações, monografias, etc).\\\hline
                                                         & Estudos não escritos em inglês.\\\hline
                                                        
        
        \end{tabular}
        \legend{Fonte: Elaborado pelo autor}
        \end{table}
  \begin{comment}
  
  \section{Critérios de avaliação de qualidade}
        
        A avaliação da qualidade é crítica em uma revisão sistemática para investigar se diferenças de qualidade nos artigos selecionados fornecem uma explicação para as diferenças nos resultados obtidos \cite{kitchenham}. Este trabalho considera que a qualidade se relaciona com a medida em que o estudo minimiza o viés e maximiza a validade e credibilidade, por meio dos critérios descritos na Tabela \ref{tab:quality}.
        
       
\begin{table*}[h!]
 \centering
     \scriptsize
 \caption{Critérios de avaliação da qualidade do estudo.}
 \begin{tabular}{p{10cm}p{0.5cm}p{0.5cm}p{0.5cm}p{0.5cm}p{0.5cm}}
 \hline
 \textbf{Critério} & \textbf{Eva}	& \textbf{Val} & \textbf{Sol} & \textbf{Exp} & \textbf{Op}\\
 \hline
 Q1. Possui uma declaração clara dos objetivos da pesquisa? &x	&x	&x	&x\\
 \hline
Q2. A técnica proposta é  descrita claramente? & & &x	& \\
 \hline
 Q3. Há uma descrição adequada do contexto (indústria, ambiente de laboratório, produtos utilizados e assim por diante) em que a pesquisa foi realizada? 	&x	&x	& & \\
 \hline	





Q1. Existe discussão sobre os resultados do estudo? 	&x	&x	&x &\\
 \hline	
Q2. As limitações do artigo são explicitamente discutidas? &x	&x	&x	&\\
 \hline
Q3. As lições aprendidas são interessantes? & & & &x\\
 \hline
Q4.  O artigo é relevante para os profissionais da indústria?	&x	&x	&x	&x\\
 \hline
Q5. Há discussão suficiente sobre trabalhos relacionados? (As técnicas concorrentes são discutidas e comparadas com a técnica atual?)	&x	&x	&x	&\\
 \hline
Q6. Os participantes do estudo ou unidades de observação são descritos adequadamente? Por exemplo, experiência em Engenharia de Software, tipo (estudante, profissional, consultor), nacionalidade, experiência em tarefas e outras variáveis relevantes. &x	&x & &\\
 \hline		

Q7. O estudo amplia suficientemente o conhecimento sobre comunicação de requisitos no processo de desenvolvimento de software?	&x	&x	&x	&x\\
 \hline	
Q8. O posicionamento sobre o tema é adequado?	&	&	&	& & x\\
 \hline	
Q9. É provável que provoque discussão sobre o tema?	&	& &  &	&x\\
 \hline	

Q10. Quão claras são as hipóteses/concepções teóricas/valores que moldaram as configurações e as opiniões descritas?	&	&	&	& & x\\
 \hline
 \end{tabular}%
 \legend{Fonte: Adaptado de \cite{vilela2017integration}.}
 \label{tab:quality}%
\end{table*}%

        A avaliação da qualidade dos artigos foi realizada por meio de uma técnica de pontuação para avaliar a credibilidade, completude e relevância dos estudos selecionados. Os artigos serão avaliados em relação a um conjunto de 10 critérios de qualidade proposto por \cite{vilela2017integration}.
        
        O trabalho diferencia os estudos em cinco categorias: artigos de avaliação (\emph{Evaluation Research Papers} - EVA); artigos de validação (\emph{Validation Research} Papers - VAL); Propostas de solução (\emph{Solution Proposal Papers} - SOL); Relatos de experiência (\emph{Experience Papers} - EXP); e artigos de opinião (\emph{Opinion Papers} - OP). %Os critérios são descritos na Tabela \ref{tab:quality}.
        
        Cada critério da avaliação de qualidade possui três respostas possíveis: ``Sim'' (pontuação = 1), ``Parcialmente'' (pontuação = 0,5) ou ``Não'' (pontuação = 0). Consequentemente, o grau de qualidade de um artigo é calculado considerando a soma das pontuações das respostas para as questões relacionadas ao seu tipo de pesquisa.
\end{comment}  
   
\section{Formulário de extração de dados}
    
     Para guardar todas as informações necessárias para responder às perguntas da pesquisa foi preparado o formulário de extração de dados apresentado na Tabela \ref{tab:extraction}.
    
    \begin{table}[h]
    \centering
    \scriptsize
    \caption{Formulário de extração dos dados.}
    \label{tab:extraction}
    \begin{tabular}{p{5cm}p{6cm}p{4cm}}
    \hline
    \textbf{Dado} & \textbf{Descrição}  & \textbf{Pergunta de Pesquisa}\\ \hline
    Autores, ano, título     &  &Visão Geral dos estudos  \\ \hline
    Origem do trabalho       & IEEE, ACM, Springer, Scopus, Science Direct    \\ \hline
  
   Stakeholders envolvidos & &P1  \\ \hline
    
    Fator  de comunicação   & &P2 \\ \hline
    
    Impacto do fator de comunicação & &P3 \\ \hline
   
    Aspecto do processo de comunicação afetado & &P4 \\ \hline
   
    Canal de comunicação utilizado & &P5 \\ \hline
   
  
    \end{tabular}
    \legend{Fonte: Elaborado pelo autor.}
    \end{table}
        
\section{Procedimento para seleção de estudos}

        O procedimento de seleção de estudos consistiu em cinco etapas principais. Na primeira, os estudos foram consultados e obtidos por meio de busca automática usando a \emph{string} de pesquisa nas bases de dados apresentadas na Tabela \ref{tab:bases}. Os resultados das buscas foram armazenados na ferramenta Parsifal.

        A segunda etapa consistiu na eliminação de artigos duplicados. Em seguida, no passo 3, houve a seleção dos estudos primários obtidos na etapa anterior por meio da leitura de título e \emph{abstract} usando os critérios de inclusão e exclusão descritos na Seção \ref{sec:criterios}. Se houve dados insuficientes ou dúvidas, o artigo seguiu para o próximo passo.
        
        O quarto passo consistiu na leitura completa dos artigos para responder as perguntas de pesquisa usando o formulário de extração da Tabela \ref{tab:extraction}.
      
    \section{Ameaças à validade} 
        
        A classificação de ameaças à validade descrita por \cite{Wohlin2000} foi utilizada para discutir as ameaças deste trabalho. Esta classificação define quatro tipos de ameaças de validade, sendo elas, ameaças de conclusão, internas, de construção e de validade externa.
        
        \emph{Ameaças à validade de constructo:} Este tipo de validade diz respeito à generalização do resultado para o conceito ou teoria por trás da execução do estudo. Com o objetivo de minimizar ameaças dessa natureza, foi utilizado sinônimos para as principais palavras chaves.
        
       \emph{Ameaças de validade interna:} estão relacionadas a uma possível conclusão errada sobre as relações causais entre o tratamento e o resultado \cite{Wohlin2000}. Decisões subjetivas podem ocorrer durante a seleção de artigos e extração de dados uma vez que é comum estudos primários não fornecerem uma descrição clara ou objetivos e resultados apropriados, dificultando a aplicação objetiva dos critérios de inclusão/exclusão ou a imparcialidade. A fim de minimizar erros de seleção e extração, o processo de seleção foi realizado de forma iterativa de forma que quando ocorreu dúvida na aplicação de algum critério, o estudo não foi eliminado e passou para a próxima fase. Além disso, o processo de seleção foi realizado de forma colaborativa pelo autor e por um colaborador de forma que conflitos foram discutidos e solucionados pelos mesmos em conjunto com a orientadora. Dessa forma, objetivou-se atenuar as ameaças devido ao viés pessoal na compreensão do estudo.
        
        \emph{Ameaças à validade externa:} está relacionada ao grau em que os estudos primários serão representativos para o tópico de revisão. Se tratando de uma revisão da literatura, a validade externa depende da literatura identificada: se a literatura identificada não é valida externamente, tampouco a síntese do seu conteúdo é citada. Esta ameaça foi mitigada devido a utilização do critério de exclusão para eliminar, da pesquisa, os estudos provenientes de literatura cinza. Além disso, para mitigar as ameaças externas, o protocolo de pesquisa foi definido iterativamente e, validado, com o consenso do autor, do colaborador e da orientadora.  
        
        \emph{Ameaças à validade de conclusão}: A metodologia descrita por \cite{kitchenham} assume que nem todos os estudos primários relevantes possam vir a ser identificados. Para amenizar os efeitos dessa ameaça, o processo da revisão foi cuidadosamente elaborado e discutido pelos autores para minimizar o risco de exclusão estudos relevantes. Outro método empregado foi utilizar expressões e palavras sinônimas para os constructos dessa revisão sistemática, essa técnica objetiva uma maior cobertura de estudos possivelmente importantes a partir da pesquisa automática. Além disso, o processo de seleção do estudo foi conduzido em paralelo e de forma independente pelo aluno e pelo colaborador. Posteriormente, os resultados foram harmonizados para mitigar o viés pessoal na seleção do estudo causado por revisores individuais. Finalmente, a orientadora supervisionou esse processo.
        
        Para minimizar ameaças no processo de seleção dos estudos calculou-se um índice de concordância na etapa de classificação de artigos, feito junto a um colaborador e o autor do estudo, Os resultados obtidos foram satisfatórios visto que o menor número de concordância obtido foi 85\% e estão ilustrados na Tabela \ref{tab:concord}.
        
        
        \begin{table}[h!]
        \caption{Índice de concordância}
        \label{tab:concord}
\begin{tabular}{|l|l|l|l|l|l|}
\hline
\textbf{Base} & \textbf{Artigos} & \textbf{Duplicados} & \textbf{Autor Rejeitou} & \textbf{Colaborador Rejeitou} & \textbf{Concordância} \\ \hline
ACM           & 8                & 3                   & 4                      & 4                           & 100\%                 \\ \hline
IEEE          & 211              & 17                  & 161                    & 163                         & 92,2\%                \\ \hline
SCIENCE       & 324              & 16                  & 274                    & 255                         & 85,7                  \\ \hline
SCOPUS        & 317              & 41                  & 241                    & 220                         & 90,4                  \\ \hline
\end{tabular}
 \legend{Fonte: Elaborado pelo autor}
\end{table}
        
        
        
        
O próximo capítulo apresenta a condução da revisão sistemática.      

\newpage

\begin{comment}ou mapeamento sistematico
%bolsista
\begin{alineascomponto}
    \item Escolha de trabalhos;
    \item Definição dos critérios;
    \item Identificação se existe algum fator;
    \item Classificação dos fatores;
    \item Construção do modelo de classificação;
    \item Validação do modelo;

\end{alineascomponto}

%, lembrando que alguns passos serão executados em duas partes, duas pessoas diferentes para obtenção de dados mais confiáveis e maior credibilidade.

\section{Escolha de trabalhos}

Para que um trabalho seja escolhido e relacionado será feita uma primeira triagem selecionando entre trabalhos relacionados á desenvolvimento de software e coletando os que possuem palavras-chave que estejam diretamente relacionadas com o tema, abrangendo também outras palavras chaves que poderão ser úteis para a coleta dos fatores mesmo que no devido trabalho não haja um relacionamento explícito.

Palavras chaves diretas:
\begin{itemize}
    \item Comunicação
    \item Problemas de comunicação 
    \item Otimização de comunicação
    \item Comunicação de requisitos
    \item Engenharia de requisitos 
    \item Artefatos de requisitos
    \item Boas práticas
\end{itemize}

Nos trabalhos com as seguintes palavras chaves será buscado se há relacionamento mesmo que indireto com comunicação

Palavras chaves de relacionamento indireto:
\begin{itemize}
    \item Melhoria de processo 
    \item Metodologias de otimização
    \item Levantamento de boas práticas
\end{itemize}


\subsection{Definição dos critérios}

Os critérios de seleção são descritos como algo que afeta diretamente ou indiretamente a comunicação de requisitos independente do que se trata.

Critérios:

\begin{itemize}
    \item Afeta positivamente a comunicação ?
    \item Afeta negativamente a comunicação ?
    \item Otimiza o processo de comunicação ?
    \item Ocasiona ruído de comunicação ? 
    \item Descreve prática relacionado a comunicação ?
    \item Descreve metodologia relacionado a comunicação ?
\end{itemize}

\section{Identificação se existe algum fator}

Com os critérios de seleção definidos iremos verificar se o trabalho selecionado  descreve algum problema, melhoria, otimização ou identifica algo que está relacionado direto ou indiretamente a comunicação inserido no contexto do desenvolvimento de software, se sim iremos decidir se  tal informação é um fator que pode influenciar no processo de comunicação, ressaltando que para a informação seja selecionada é necessário que atenda a um ou mais dos critérios descritos na etapa anterior.

\section{Classificação dos fatores}

Com os fatores selecionados será feito então uma análise individual entre os fatores para identificar se a ocorrência do mesmo auxilia, impede, melhora ou piora a comunicação. 

\section{Avaliação de qualidade} 

A avaliação da qualidade é crítica em uma revisão sistemática para investigar se diferenças de qualidade nos artigos selecionados fornecem uma explicação para as diferenças nos resultados obtidos \cite{kitchenham}. Seguindo as diretrizes desses autores, este trabalho considera que a qualidade se relaciona com a medida em que o estudo minimiza o viés e maximiza a validade interna e externa.

A avaliação da qualidade dos artigos selecionados será realizada por meio de uma técnica de pontuação para avaliar a credibilidade, completude e relevância dos estudos selecionados. Os artigos serão avaliados em relação a um conjunto de 20 critérios de qualidade proposto por \cite{vilela2017integration}.

O trabalho de diferencia os estudos em cinco categorias: artigos de avaliação (Evaluation Research Papers - EVA); artigos de validação (Validation Research Papers - VAL); Propostas de solução (Solution Proposal Papers - SOL); Relatos de experiência (Experience Papers - EXP); e artigos de opinião (Opinion Papers - OP). Os critérios são descritos na Tabela \ref{tab:quality}.

Cada critério da avaliação de qualidade possui três
respostas possíveis: ``Sim" (pontuação = 1), ``Parcialmente'' (pontuação = 0,5) ou ``Não'' (pontuação = 0). Consequentemente, o grau de qualidade de um artigo é calculado considerando a soma das pontuações das respostas para as questões relacionadas ao seu tipo de pesquisa. A maior pontuação que um artigo poderá obter será 100\% ou seja 20 pontos, e a pontuação mínima para o artigo ser analisado nessa revisão sistemática será de 50\%, ou seja 10 pontos.

\begin{table*}[h!]
 \centering
     \scriptsize
 \caption{Critérios de avaliação da qualidade do estudo.}
 \begin{tabular}{p{10cm}p{0.5cm}p{0.5cm}p{0.5cm}p{0.5cm}p{0.5cm}}
 \hline
 \textbf{Critério} & \textbf{Eva}	& \textbf{Val} & \textbf{Sol} & \textbf{Exp} & \textbf{Op}\\
 \hline
 Q1. Possui uma declaração clara dos objetivos da pesquisa? &x	&x	&x	&x\\
 \hline
Q2. A técnica proposta é  descrita claramente? & & &x	& \\
 \hline
 Q3. Há uma descrição adequada do contexto (indústria, ambiente de laboratório, produtos utilizados e assim por diante) em que a pesquisa foi realizada? 	&x	&x	& & \\
 \hline	
Q4. Os tratamentos foram alocados aleatoriamente? &	x & & & 	\\
 \hline		
Q5. A amostra é representativa do público para a qual os resultados serão generalizados? &x	&x	& & \\
 \hline	
Q6. Havia algum grupo de controle presente com o qual os tratamentos pudessem ser comparados, se aplicáveis?	&x	& & &	\\
 \hline	
Q7. Caso tenha um grupo de controle, os participantes são semelhantes aos do grupo de tratamento em termos de variáveis que podem afetar os resultados obtidos? 	&x	& & &			\\
 \hline
Q8. A análise de dados foi rigorosa o suficiente?	&x	&x	& & \\
 \hline	
Q9. Existe discussão sobre os resultados do estudo? 	&x	&x	&x &\\
 \hline	
Q10. As limitações do artigo são explicitamente discutidas? &x	&x	&x	&\\
 \hline
Q11. As lições aprendidas são interessantes? & & & &x\\
 \hline
Q12.  O artigo é relevante para os profissionais da indústria?	&x	&x	&x	&x\\
 \hline
Q13. Há discussão suficiente sobre trabalhos relacionados? (As técnicas concorrentes são discutidas e comparadas com a técnica atual?)	&x	&x	&x	&\\
 \hline
Q14. Os participantes do estudo ou unidades de observação são descritos adequadamente? Por exemplo, experiência em Engenharia de Software, tipo (estudante, profissional, consultor), nacionalidade, experiência em tarefas e outras variáveis relevantes. &x	&x & &\\
 \hline		
Q15. Existe evidências de cuidados com as considerações éticas?	&x	&x	& &	\\
 \hline
Q16. O estudo amplia suficientemente o conhecimento sobre ferramentas colaborativas no desenvolvimento do software?	&x	&x	&x	&x\\
 \hline	
Q17. O posicionamento sobre o tema é adequado?	&	&	&	& & x\\
 \hline	
Q18. É provável que provoque discussão sobre o tema?	&	& &  &	&x\\
 \hline	
Q19. Quão bem a diversidade de perspectiva e o contexto foram explorados? 	&	&	&	& & x\\
 \hline	
Q20. Quão claras são as hipóteses/concepções teóricas/valores que moldaram as configurações e as opiniões descritas?	&	&	&	& & x\\
 \hline
 \end{tabular}%
 \label{tab:quality}%
\end{table*}%

\section{Procedimento para seleção de estudos}

O procedimento de seleção de estudos, apresentado na Figura consistirá em cinco etapas principais. Na primeira, os estudos serão consultados e obtidos por meio de busca automática usando a \emph{string} de pesquisa nas bases de dados apresentadas na Tabela \ref{tab:bases}. Os resultados das buscas serão armazenados na ferramenta Parsifal.

A segunda etapa consistirá na eliminação de artigos duplicados. Em seguida, no passo 3, haverá a seleção dos estudos primários obtidos na etapa anterior por meio da leitura de título e \emph{abstract} usando os critérios de inclusão e exclusão descritos na Seção \ref{sec:criterios}. Se houver dados insuficientes ou dúvidas, o artigo seguirá para o próximo passo.

O quarto passo consistirá na leitura completa dos artigos para responder as perguntas de pesquisa descritas na Seção \ref{sec:perguntas} usando o formulário de extração da Tabela \ref{}.

Finalmente, será realizada a avaliação da qualidade dos artigos usando os critérios da Tabela \ref{tab:quality}.

\section{Ameaças à validade}

We used the categorization of threats presented by Wholin et. al \cite{Wohlin2000}, which includes four types of validity threats, namely, conclusion, internal, construct, and external validity threats. 

\emph{Construct validity}: Construct validity is related to generalization of the result to the
concept or theory behind the study execution \cite{Wohlin2000}. Aiming to minimize threats of this nature, we used many synonyms for the main constructs in this review: ``safety-critical systems'', ``requirements engineering'', ``safety requirements'', and ``communication''. 

\emph{Internal validity} threats are related to possible wrong conclusion about causal relationships between treatment and outcome \cite{Wohlin2000}. The primary objective of conducting a SLR is to minimize internal validity threats in the research. Decisões subjetivas podem ocorrer durante a seleção de artigos e extração de dados uma vez que é comum estudos primários não fornecerem uma descrição clara ou objetivos e resultados apropriados, dificultando a aplicação objetiva dos critérios de inclusão/exclusão ou a imparcialidade. A fim de minimizar erros de seleção e extração, o processo de seleção será realizado de forma iterativa de forma que quando ocorrer dúvida na aplicação de algum critério, o estudo não será eliminado e passará para a próxima fase. Além disso, o processo de seleção será realizado de forma colaborativa pela aluno e por um colaborador de forma que conflitos sejam discutidos e solucionados pelos alunos em conjunto com a orientadora. Dessa forma, objetiva-se atenuar as ameaças devido ao viés pessoal na compreensão do estudo.

\emph{External validity} is concerned with establishing the generalizability of the SLR results, which is related to the degree to which the primary studies are representative for the review topic. In the case of a literature review, the external validity depends on the identified literature: if the identified literature is not externally valid, neither is the synthesis of its content \cite{gasparic2016recommendation}. Esta ameaça será mitigada devido a utilização do critério de exclusão para eliminar, da pesquisa, os estudos provenientes de literatura cinza. Além disso, para mitigar as ameaças externas, o protocolo de pesquisa foi definido iterativamente e, validado, com o consenso da autora, do colaborador e da orientadora. Esta ameaça será mitigada devido a utilização do critério de exclusão para eliminar, da pesquisa, os estudos provenientes de literatura cinza. Além disso, para mitigar as ameaças externas, o protocolo de pesquisa foi definido iterativamente e, validado, com o consenso da autora, do colaborador e da orientadora.

\emph{Conclusion validity}: The used methodology of Kitchenham and Charters \cite{kitchenham} already assumes that not all relevant primary studies that exist can be identified. nem todos os estudos primários que existem  relacionados à pesquisa podem ser identificados, sendo assim, para minimizar esse ameaça, o processo da revisão foi cuidadosamente elaborado e discutido pelos autores para minimizar o risco de exclusão
estudos relevantes. Outro método empregado foi utilizar expressões e palavras sinônimas para os constructos dessa revisão sistemática, essa técnica objetiva uma maior cobertura de estudos possivelmente importantes a partir da pesquisa automática. Além disso, o processo de seleção do estudo será conduzido em paralelo e de forma independente entre autor e pelo colaborador. Posteriormente, os resultados serão harmonizados para mitigar o viés pessoal na seleção do estudo causado por revisores individuais. Finalmente, a orientadora supervisionará esse processo.

\end{comment}