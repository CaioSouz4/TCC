\chapter{Trabalhos Relacionados}
\label{cap:trabalhos-relacionados}



A comunicação de requisitos no contexto de desenvolvimento de software é um tema investigado em diversos trabalhos \cite{liskin2015artifacts, Maquison, coughlan}. A Tabela 1 apresenta as principais semelhanças e diferenças entre este trabalho e os trabalhos analisados.    

\begin{table}[h]
\centering
\caption{\text{Comparação entre os trabalhos relacionados e o proposto.}}
\begin{center}
\scriptsize
\label{diferenca_trabalhos}

\begin{tabular}{|p{2.5cm}|p{2.2cm}|p{2.0cm}|p{2.0cm}|p{2.0cm}| }
\hline
\rowcolor{} & & & & 
\\
\rowcolor{} \textbf{Critério}&\textbf{\cite{liskin2015artifacts}}&\textbf{\cite{coughlan}}&\textbf{\cite{Maquison}} &\textbf{Trabalho Proposto} \\

\hline
& & & &  \\
Foco de análise &Artefatos de requisitos. & Problemas na comunicação. &Desafios na comunicação. &Todo o processo de desenvolvimento.  \\


\hline
& & & & \\
Fase do processo de desenvolvimento &Todo o processo de desenvolvimento. &Todo o processo de desenvolvimento. &Todo o processo de desenvolvimento &Todo o processo de desenvolvimento. \\
& & & & \\
\hline
& & & & \\
Metodologia de coleta de dados &Entrevista com 21 desenvolvedores. &Entrevistas semi-estruturadas com 5 participantes. &Survey com 24 respostas. &Revisão sistemática. \\
\hline


\end{tabular} 
\legend {\fontsize{10}{12}\selectfont {Fonte: Elaborado pela autor.}}
\end{center}
\end{table}

%Nesta seção, serão apresentado trabalhos que possuem relacionamento com este trabalho, o primeiro trabalho destacado trata-se do trabalho da \citeonline{liskin2015artifacts} que  contém um alto relacionamento com o direcionamento da pesquisa deste trabalho. 
    
    %qual a relação dele com seu trabalho, de que forma contribui; 2) que maneira a proposta se assemelha ao trabalho relacionado, ou seja, qual a relação direta entre os dois; 3) por fim, informa-se em que aspecto a proposta se difere do trabalho relacionado. Escreva de forma fluente, de maneira que não se perceba três fragmentos no texto.
    
     \citeonline{liskin2015artifacts} investiga como profissionais de diversos cargos usam artefatos de requisitos, como eles conseguem trabalhar com vários artefatos e se utilizam práticas atuais para vincular artefatos relacionados. 
    
    A investigação ocorreu por meio de entrevistas com 21 profissionais de 6 empresas. As entrevistas indicaram que, muitas vezes, é necessária uma variedade de tipos de artefatos para conduzir com sucesso um projeto. Ao mesmo tempo, o uso de vários artefatos ocasiona problemas como esforço na tradução conversão e inconsistências.
    
    O trabalho de  \citeonline{liskin2015artifacts} relaciona-se diretamente com o objetivo deste trabalho, pois o mesmo descreve como artefatos impedem e auxiliam a comunicação de requisitos.     A identificação de fatores que influenciam a comunicação de requisitos é um ponto em comum, porém o foco é comunicação por meio de artefatos enquanto que neste trabalho serão investigados vários canais de comunicação (ver Seção \ref{sec:fund_comunicacao}). Além disso,  a metodologia de pesquisa diferencia-se uma vez que Liskin realiza entrevistas em empresas e, neste trabalho, será realizada uma revisão sistemática da literatura (ver Seção \ref{sec:fund_slr}). 
  
   Em \citeonline{Maquison}, são identificadas boas práticas e desafios que ocorrem  na  elicitação  de  requisitos  em  empresas  de  desenvolvimento  de  software. Esse levantamento é realizado por meio de um \emph{survey} respondido por 24 profissionais de empresas  de  desenvolvimento  de  software. O trabalho de \citeonline{Maquison} relaciona-se com este estudo mesmo que de uma forma indireta e breve, pois o mesmo realiza um levantamento de boas práticas e desafios na elicitação de requisitos de software. 
   
   A elicitação é uma das fases da engenharia de requisitos que é a etapa inicial do processo de desenvolvimento de software. Sendo assim, o objetivo de \citeonline{Maquison} assemelha-se a este trabalho ao abordar no seu estudo a comunicação na área de requisitos. Portanto, a principal diferença trata-se de que este trabalho objetiva analisar fatores no processo de comunicação ao longo do desenvolvimento de software enquanto que \citeonline{Maquison} aborda desafios e boas práticas na elicitação de requisitos.
    
    Finalmente, problemas de comunicação na elicitação de requisitos são analisados no trabalho de \citeonline{coughlan} por meio de uma análise de entrevistas semi-estruturadas com 7 participantes (dois gerentes de projeto e cinco consultores de negócios). O estudo desses autores assemelha-se bastante, pois realiza um estudo diretamente nos problemas de comunicação. 
    
   A pesquisa de \citeonline{coughlan} procura apresentar uma estrutura de comunicação sobre o processo de engenharia de requisitos. Os autores destacam o fato de que os problemas de comunicação estão diretamente inclusos nos muitos problemas que possam vir a ocorrer na engenharia de requisitos. Sendo que estes consistentemente implicam em muitas falhas de projeto. A principal diferença da pesquisa para este é foco investigado, sendo que analisa problemas de comunicação que venham a ocorrer e não apresenta boas práticas ou como os fatores se relacionam. 
   
   A partir da identificação de semelhanças e diferenças com estudos relacionados, os objetivos desse trabalho foram definidos conforme descrito no próximo capítulo.

