\chapter{Introdução}
\label{cap:introducao}



    Em toda sua abrangência, o mundo moderno depende de sistemas computacionais \cite{sommerville}. Estes estão presentes nos mais variados setores e domínios como, por exemplo, transportes, saúde, agricultura, educação, entre outros e fornecem suporte a automatização de tarefas.
    
    A engenharia de software surgiu a partir da necessidade de desenvolvimento de sistemas complexos ao proporcionar metodologias e boas práticas na produção de software \cite{sommerville}. As metodologias definem, de forma geral, cinco etapas para o desenvolvimento de software: requisitos, arquitetura, implementação, teste e manutenção \cite{sommerville, pressman2005software}.
    
A demanda por processos de desenvolvimento de softwares cada vez mais rápidos e eficientes é evidente no mercado de software \cite{pressman2005software}. Um fator preponderante na melhoria desse aspecto importantíssimo no desenvolvimento é um processo de engenharia de requisitos eficaz \cite{kotonya}.

A engenharia de requisitos é o processo de compreensão e definição dos serviços requisitados do sistema \cite{sommerville}. Esta fase é de suma importância nos processos de desenvolvimento uma vez que seu objetivo é realizar o levantamento das necessidades dos \emph{stakeholders} bem como especificar o sistema de forma que os requisitos sejam satisfeitos.

A fase de engenharia de requisitos é a mais crítica uma vez que uma grande proporção (48\%) de problemas no desenvolvimento \cite{hall2002requirements} são derivados de problemas com os requisitos \cite{unirepm, gorschekREModel}. Corrigir esses problemas relacionados a requisitos tem um alto custo devido ao retrabalho em fases posteriores \cite{boehm1988costs, leffingwell1997calculating}.

Quando a especificação do sistema é construída de forma apurada, ela auxilia no desenvolvimento e na manutenção do software. \cite{kotonya} destaca que estima-se que o custo de correção de um erro de requisitos possa ser de até 100 vezes o custo de corrigir um erro simples de programação.

Um dos aspectos no processo de engenharia de requisitos essenciais para sua eficácia é a comunicação. A comunicação de requisitos é o poder de transmitir informações dos requisitos, seja entre artefatos ou de artefatos para desenvolvedores \cite{liskin2015artifacts}. A comunicação contribui para que as informações fluam sem conflitos em toda a extensão do projeto \cite{Peixoto}. Portanto, garantir uma boa comunicação proporciona flexibilidade e rapidez no processo \cite{Peixoto}.

Existem fatores que influenciam, positivamente e negativamente, a comunicação de requisitos em diferentes etapas do processo \cite{coughlan, liskin2015artifacts}. \citeonline{liskin2015artifacts} identificou que os artefatos de requisitos contribuem para encontrar informações rapidamente, sendo assim, uma característica positiva atrelada a um artefato. Por outro lado, \citeonline{Maquison} constatou que a existência de rivalidades e animosidades na equipe resultam na destruição de canais de comunicação formais na etapa de elicitação de requisitos tendo alto impacto negativo no projeto.

Sendo assim, o estudo de fatores que impactam a comunicação de requisitos é de fundamental importância para obter melhores resultados no desenvolvimento de software. É importante ressaltar que proporcionar melhorias na comunicação em todo o processo é uma constante prioridade das empresas que prezam por processos de boa qualidade \cite{sommerville}.

Algumas técnicas já foram mapeadas para tratar e melhorar desafios da comunicação de requisitos \cite{jdavis, Stapel}. Entretanto, foram encontrados poucos trabalhos que realizem um levantamento de boas práticas e fatores que influenciam negativamente a comunicação de requisitos ao longo do processo de desenvolvimento de software.

Este trabalho visa levantar e propor um modelo de classificação dos fatores influentes na comunicação de requisitos ao longo do processo de desenvolvimento de software, dando ênfase aos artefatos de requisitos. 

As informações a serem utilizadas para construir o modelo serão obtidas por meio de revisão sistemática da literatura \cite{kitchenham}, que consiste em realizar busca em diferente bases de dados e filtrar estudos relevantes sobre um determinado assunto a partir de critérios estabelecidos. O objetivo é coletar o máximo possível de informações sobre os fatores e utilizar como insumo para a construção do modelo.

A compreensão e análise desses fatores que afetam a comunicação de requisitos é determinante para obter um processo de desenvolvimento eficaz. Sendo assim, o conhecimento dos fatores é relevante para a indústria, pois permite que os gerentes de projetos e analistas elaborem estratégias para mitigar os fatores negativos e maximizem os efeitos dos fatores positivos. 

Além disso, auxilia no contexto organizacional de empresas maduras que possuem melhor capacidade em atender vários clientes, mas falham em atender com excepcional satisfação como as organizações pequenas. Estas também podem se beneficiar ao otimizar o processo de desenvolvimento em um escopo abrangente ou uma demanda maior de clientes \cite{laukkanen2018comparison}.

Finalmente, este trabalho proporciona benefícios a academia uma vez que poderá ser utilizado como uma agenda de pesquisa que consistiria na realização de estudos mais detalhados em trabalhos futuros.

Este trabalho possui a seguinte estrutura. No Capítulo 2, são apresentados trabalhos relacionados, que tratam de estudos na área de comunicação de requisitos. O Capítulo 3 apresenta os objetivos gerais e específicos do trabalho. No Capítulo 4, os principais conceitos necessários ao entendimento deste trabalho são discutidos. O Capítulo 5 descreve a metodologia utilizada; no Capítulo 6 é apresentado o protocolo da revisão sistemática; o Capítulo 7 apresenta os resultados da revisão sistemática e o Capítulo 8 apresenta conclusões e trabalhos futuros.
