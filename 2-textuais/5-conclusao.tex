\chapter{CONCLUSÃO E TRABALHOS FUTUROS}
\label{chap:conclusoes-e-trabalhos-futuros}



A fase de engenharia de requisitos é a mais crítica uma vez que uma grande proporção de problemas no desenvolvimento são derivados de problemas com os requisitos. Um dos aspectos no processo de engenharia de requisitos essenciais para sua eficácia é a comunicação. Esta contribui para que as informações fluam sem conflitos em toda a extensão do projeto. Portanto, garantir uma boa comunicação proporciona flexibilidade e rapidez no processo.

Existem fatores que influenciam, positivamente e negativamente, a comunicação de requisitos em diferentes etapas do processo. Sendo assim, o estudo de fatores que impactam a comunicação de requisitos é de fundamental importância para obter melhores resultados no desenvolvimento de software. Nesse contexto, proporcionar melhorias na comunicação em todo o processo é uma constante prioridade das empresas que prezam por processos de boa qualidade.

Este trabalho teve como objetivo identificar fatores que interferem na comunicação de requisitos ao longo do desenvolvimento de software, Para alcançar tal meta, uma revisão sistemática da literatura foi realizada. A revisão retornou 860 trabalhos dos quais 94 foram aceitos para extração.

Sendo assim, este trabalho possibilitou responder as seguintes perguntas de pesquisa:

\textbf{P1: Quais são os stakeholders envolvidos?} observou-se um deficit de comunicação nos papéis de clientes, engenheiro de requisitos e desenvolvedores.

\textbf{P2: Quais são os fatores de comunicação apontados no estudo?} este trabalho identificou um total de 38 fatores distribuídos em positivos e negativos.

\textbf{P3: Quais são os problemas de comunicação reportados no estudo?}, foi identificado 16 fatores de comunicações negativos que ocasionam problemas na comunicação. 

\textbf{P4: Qual o impacto do fator de comunicação?} a gravidade dos fatores se mostrou critica devido ao fato de influenciarem na distorção e oclusão de informações ameaçando assim a qualidade do projeto. Os fatores foram classificados como grave e médio, tendo uma distribuição de 60,64\%  e 39,36\% respectivamente.

\textbf{P5: O fator interfere em qual aspecto do processo de comunicação} observou-se que fatores positivos geralmente estão relacionados a canais de comunicação eficientes, e negativos estão relacionados a conexão emissor/receptor.

A análise mostrou também que a ocorrência dos fatores em mais de um trabalho contribuindo para a qualidade dos dados obtidos. O trabalho contribui de maneira que evidencia a existência das questões de comunicação no desenvolvimento de software, identificando fatores e a natureza de suas origens assim como a atuação e o impacto dos mesmos no processo de comunicação.

Como trabalhos futuros, sugere-se um estudo detalhado sobre os fatores identificados e sobre o modelo de classificação dos fatores assim possíveis trabalhos futuros podem ser:

\begin{itemize}
    \item Validar o modelo de classificação deste estudo por profissionais de desenvolvimento de software.
    \item Investigar formas de mitigação das falhas humanas e organizacionais de comunicação.
    \item Validar o grau em que os fatores positivos descritos realmente contribuem no repasse de informações de requisitos.
    \item Avaliar a qualidade dos estudos selecionados neste trabalho.

\end{itemize}


 \newpage 
